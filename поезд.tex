00:00:19 — У нас чай, кофе, если что, мы же на 7 дней набрали, надо вам что-то? Нет, спасибо. Ну, не особо. Смотрите, чай у нас есть.
00:00:28 — У меня чай есть. Я там чисто классички купил. Спасибо большое. В Питере вы не были, да? В Питере? А, был. В Питере был, да. В Москве? В Москве не был. В Питере я был, когда с Парижа возвращался. Маленько задержался.
00:00:56 — Пару дней. В Питер? В Питер, ну… Тяжело сказать. Насколько я его видел, ну, мне как-то он показался каким-то как центр, каким-то слишком таким пустынным каким-то.
00:01:36 — Пустынным? Пустынным, не знаю, слишком масштабным каким-то таким разбором.
00:01:56 — Большой, вот нерационально, то есть слишком как бы размашисто.
00:02:01 — Не оптимально. Большая площадь. Площадь огромная, да?
00:02:10 — Да, площадь огромная. Если бы это вместить всё, там, допустим, построить небоскрёб, туда бы это всё влезло, в один небоскрёб.
00:02:49 — А вот это как раз чувство, когда тебе всё нравится, это, мне кажется, слово, вот именно любовь как общее охватывающее понятие, а когда тебе что-то, ты чем-то недовольны, то это как раз вот отсутствие любви, и тяжелее всего именно любить, когда тебе нет ответа на вот это чувство вознаграждения, что ты что-то делаешь, а тебе результат в ответ.
00:03:23 — Так легко любить, а когда результата нет, тогда тяжелее сохранить вот это отношение. И вот как раз вот эти все говорят, что кто святые, да, они несмотря ни на что, они вот могут сохранить чувство, даже если оно не вызывает никакой ответа, реакции, ответа.
00:03:49 — То есть это как-то надо внутри себя, то есть создавать самому. А вы в Бога верите? Меня как бабушка говорила. Она говорила так. Я говорю, не верю, но что-то есть. Что-то есть. Что-то выше нас. Крестика не носите? Нет, не ношу. Но я крещу. Ну да, в церкви крестили, да.
00:04:20 — Что-то есть выше нас. О, так это я это, съел чьё-то, у меня такие же на столе лежали, я думал это моё, а то из коробочки ещё, а я схватил, а я не знаю, на столе лежали, я думал это от моего набора.
00:04:55 — А, да, я, наверное, его схватил. Ну, ладно, я положил на столе, пусть лежат.
00:05:14 — Моё отношение именно к вере, оно такое, что по совести надо жить.
00:05:40 — Впринципе, да. Это не получится по совести.
00:05:45 — Как это? Мы как в этом. А я вам объясню. Совесть, это…
00:05:53 — Правильно? Ну да.
00:05:54 — И сниматься не жалет? Ну да. Ну да. Если разве будут обижаться, конечно, жизнь по совести будет прекрасной. Все будут скучать. Вот это и есть. Люди в себе, как говорится, племена не видят, а в чужом соломе куда-то идут. Молодец.
00:06:30 — А люди не любят больше всего, когда им говорят, что они виноваты в чем-то, они в основном любят искать виноватых других.
00:06:42 — Да, они думают, что они настоящие, ничего плохого не делают, как это ты так себя ведешь. А если разобрать человека на самом деле, поэтому для этого нужны посторонние люди, которые могут обсудить, где этот человек и этот, как оно и есть, и проголосовать.
00:07:10 — Недаром же у всех этих великих пророков, они свою жизнь хорошо не заканчивали, То распинали, то убивали, как раз потому, что они людям правду в глазах говорили, а людям не нравится, когда им правду говорят.
00:07:31 — Да, это такая жизнь, да. Просто если бы люди все были одинаковые, то это бы прогнул пылесос. Но люди разные, люди разные. Мы не знаем, что человек прожил за свою жизнь, как он жил. Может, у него была баба, мама, может, у него еще какая-то другая ситуация была, может, еще что-то вот так. Ну и все, знаешь, надо все ждукам показать.
00:08:02 — Все идеально быть не может у кого-то.
00:08:06 — Поэтому каждый человек ищет то, что ему подходит?
00:08:09 — Да.
00:08:12 — Поэтому каждый вот постоянно находится в поиске. Как знать, что вот, допустим, что ты живешь, как ты выбрал, как определить, что это истинный путь, что он среди других самый верный, а не как. Нет такого истинного пути. Именно нужно чутьем.
00:08:32 — Тут может помочь, Господи, Бог. Направить, подсказать. Это же всё, это и есть адвокат Абсолюта. Абсолют этого не понимает. К этому надо прийти. Всё всегда может измениться ими. Истина будет стать правдой.
00:08:58 — Когда люди реально начинают верить в Бога, и меняют себя к Богу, Сразу появляются совсем другие жизнедеятельности.
00:09:11 — Самое главное для меня, вера в Бога, что для меня это понятие в себе несет, это то, что человек учится быть смиренным и свое эго приглушать. То есть, как бы…
00:09:39 — Да, то есть…
00:09:40 — .
00:09:40 — Хочешь, чтобы ответить на мой вопрос? Да, иди… Неужели? Это вообще по тебе, а, Зайка? Ха-ха-ха-ха-ха-ха-ха.
00:09:53 — Хочешь сделать… …По-настоящему, знаешь?
00:09:56 — Что делаешь?
00:09:57 — Да, в ответ, тебе надо ничего делать, тебе надо защищаться Надо просто высаживаться, принять, какие есть минусы.
00:10:07 — Просто что некоторые люди могут ощущать себя как куб земли, а как раз мера она позволяет быть более смиренным более платформы что ты как бы перед богом и как бы все…
00:10:50 — Ну и плюс еще вера в Бога, она позволяет быть более спокойным в плане того, что ты знаешь, ты можешь не бояться смерти, потому что ты знаешь, что есть более веселый мир, и у тебя на душе легче становится.
00:11:31 — Потому что когда живешь и знаешь, что смерть это конец, как-то тяжело становится. Когда есть нечто большее, сразу легче.
00:11:42 — Конец смерти, да. Конец тела. Душа живет дальше. Душа может пойти в другое, как говорится. В зверя какого-нибудь или еще куда-нибудь.
00:11:59 — Ну, вообще, изначально религия сама по себе основывалась на понятии взаимодействия жизни и смерти. То есть нечто такое, чего мы не понимаем, что таинственное, такое мистическое. Именно на грани жизни и смерти отсюда все религии и выросли.
00:12:29 — А мы можем запустить, если получится, дальше за степной дорогой.
00:12:45 — Скоро. Завтра посмотрим.
00:12:51 — Не-е-е.
00:12:54 — Блин, он большой.
00:13:02 — Я вчера по Новосибирску иду, подхожу, на пешеходном переходе стою, жду, пока эти цифры красные пройдут. Машин нет, с другой стороны дороги, пока машин нет, красный свет хотя горит, парень проходит, ко мне прям проткнуют в черных очках. Начинает мне в лицо какой-то стрижок, чесанье. Я ничего не понимаю.
00:13:29 — Не лысый этот парень?
00:13:31 — Нет. Уже есть этот парень, который…
00:13:34 — Прям вот мне прям встал, вот прям нос в нос.
00:13:37 — Вот делается, да? Делает. Нет. Ну, этот все время, этот влогик, который… На Сибирске что ли?
00:13:44 — Не знаю, на Сибирску или на Сибирь. Ну, мне нравится. Я такой стою, кулак забрал в кармане. Стою спокойно, челюсти сжал, стулы мокряк. Думаю, ну всё, подёрнется тебе сразу. Ну ничего, я стою спокойно, он стоит. Светофор догорел, я спокойно пошёл, как-то расстался.
00:14:11 — Может быть, вас снимали на видеокамеру?
00:14:14 — А что-то какое-то там… Лысый какой-то бегает. Песенку какую-то. В чёрных очках какой-то.
00:14:21 — Не лысый, не помните?
00:14:22 — Нет, по-моему, чёрный у него волосы были. Попробуйте, это пародируйте.
00:14:28 — Проверяйте. Я не понял вообще, что это было. И вы попадёте в этот…
00:14:32 — В интернет.
00:14:35 — Ну, каждый люди по-разному себя видит. Поэтому вот это, такая вот, окей.
00:14:40 — Ну, я так спокойно, как это, сдержанно стою, не дергаясь, до шов и до корпуса.
00:14:47 — Вы будете получать еще эти классные видеоролики. Сейчас посмотрю.
00:00:00 — Может быть они как-то провоцируют, что если ты что-то начнёшь ему говорить, он тебя там начнёт разводить, как-будто, типа.
00:00:06 — Ну, там ещё кто-то стоит.
00:00:08 — Фух, может быть, да-да-да-да.
00:00:15 — Может вообще пострадать, может.
00:00:17 — Да, вот я чё, это, испугался. Может он, он прям подошёл, вот, прям, ну, с носу.
00:00:21 — Я думаю, может он нож порвёт.
00:00:24 — Я не знаю, стою, не знаю, что делать.
00:00:27 — С одной стороны правильно, что у него нож на кушетке
00:00:32 — ну и стулы так напрягают, что мне жалко, ну так, я на него кашусь, так думаю
00:00:37 — даю понять, что вернешься, сразу тебя здесь закопаю
00:00:50 — или псих, или просто прихолист
00:01:02 — Молодёжь так может развлекается, людей просто достаёт, как это правильно.
00:01:05 — Так вот, я говорю, сейчас их столько развелось, что они, типа, прикалываются.
00:01:10 — Ну, прикольно потом смотреть, но я бы тоже не хотела в такой ситуации оказаться.
00:01:14 — Мне бы так.
00:01:15 — Рыжий просто, на самом деле…
00:01:20 — Ну, я вам замешать, если честно.
00:01:22 — Я не говорю, если честно, я бы даже смотрела бы в рот.
00:01:32 — Я принципиально не стану спрашивать.
00:01:34 — Это все, да.
00:01:35 — Потому что они развести могут на раз.
00:01:38 — Они вот это вот даже не умеют.
00:01:42 — Капец, она в базаре.
00:01:44 — Чудо-чудо.
00:01:46 — Хорошие, понятные.
00:02:39 — По идее вообще так по-нормальному надо держать дистанцию
00:02:46 — То есть, человек к тебе приближается, ты сразу раз, даешь ему понять, что это.
00:02:52 — .
00:02:52 — .
00:02:53 — Что есть?
00:02:56 — Что?
00:02:57 — Ты посылочник не следовал?
00:02:59 — Нет.
00:03:02 — Пустой?
00:03:03 — Ты только это откинул?
00:03:06 — Да.
00:03:07 — Вот это.
00:03:11 — Давай, я тебя буду снимать.
00:03:12 — На выключку.
00:03:13 — Зачем?
00:03:59 — А так вот если общее впечатление в принципе везде одно и то же или есть
00:04:10 — Какой-то контраст.
00:04:13 — Нужная природа?
00:04:14 — Нет, вот именно жизнь людей там.
00:04:17 — Ну есть, конечно.
00:04:19 — Но как мы проехали по городам
00:04:21 — российских,
00:04:22 — то здесь же тоже все по-разному живут.
00:04:24 — Естественно, в Москве получше, а где-то там в глубинке
00:04:26 — нужны самарики. У них там зарплата
00:04:28 — пол-пятнадцати тысяч, как нам сказали.
00:04:30 — Вообще, где-то 12 тысяч,
00:04:32 — если честно.
00:04:34 — В Беларуси,
00:04:35 — ну в основном все так, ну не то что одинаковые,
00:04:38 — Естественно, многие богатые или бедные, но в основном так, рабочий класс обыкновенный.
00:04:44 — Рабочий или не рабочий класс.
00:04:50 — Я вот пока не…
00:04:52 — Ну, у нас в деревнях больше подружек нет.
00:04:55 — По художественности.
00:04:56 — Вообще, представь, если бы я была на глубинках…
00:05:01 — Я буду посовернивать с Москвой.
00:05:28 — Я тоже считал, что где я живу, там бедные люди живут.
00:05:32 — Я съездил на горный алтай, ну как бы, не, там наоборот мне показалось, что прям в шоколаде все купаются.
00:05:44 — Потому что туда в Ливане, сейчас там большая туристическая зона, там дача у всех шишек, гора.
00:05:55 — Естественно, что сдают.
00:05:58 — Там дом не могли продать, сколько-то лет назад, мне рассказывали,
00:06:02 — были 5 лет назад, за миллион, там разваленный дом какой-то
00:06:06 — на берегу Катуни.
00:06:08 — Вот в прошлом году за 25 миллионов продали.
00:06:11 — То есть цены вообще, вообще очень сильные.
00:06:14 — Во-первых, после войны, наверное, все перестарели в Украине,
00:06:18 — все на Украину ехали сюда.
00:06:19 — А сейчас по своим местам, конечно.
00:06:21 — Потому что там турист вообще очень-очень сильно прям поднялся.
00:06:31 — Бешеный целый уровень.
00:06:52 — Поехали, Максим.
00:07:10 — У нас там всё, да, получается?
00:07:20 — Да, всё.
00:07:34 — Вон туда сначала.
00:07:37 — И еще чуть-чуть.
00:07:38 — Угу.
00:07:50 — Ну все, можно спать.
00:07:54 — Спокойной ночи.
00:08:19 — Ребята, признаюсь, не учился.
00:08:23 — Нужно рассветить, раз-два-три-четыре минуты.
00:09:51 — Ну, а вот и гитара.
00:09:53 — И гитара чистая.
00:10:00 — Чистая?
00:10:02 — Чего?
00:10:07 — Ты дал мне литературу.
00:10:44 — Ну, ты куда пошла?
00:10:49 — Так это на мой мотир, на свой фотоаппарат.



——————————

Распознано с использованием https://speech2text.ru
